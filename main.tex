\documentclass{article}
\usepackage[14pt]{extsizes}
\usepackage[utf8]{inputenc}
\usepackage[russian]{babel}
\usepackage{setspace,amsmath}
\usepackage{graphicx}
\usepackage[left=20mm, top=15mm, right=15mm, bottom=15mm, nohead, footskip=10mm]{geometry}
\usepackage[T2A]{fontenc}
\usepackage[english,russian]{babel}
\usepackage[pdf]{graphviz}
\usepackage[table]{xcolor}
\usepackage{morewrites}
\usepackage{verbatim}
\usepackage{amsmath}
\usepackage{calc}
\usepackage{tikz}
\usepackage{pgfplots}
\usepackage{import}
\usepackage{xifthen}
\usepackage{pdfpages}
\usepackage{transparent}

\begin{document} % начало документа
\begin{center}
\large{Домашнее задание №1}\\
\large{Регулярные языки и конечные автоматы}\\
\hfill \break
\hfill \break
\hfill \break
\hfill \break
\hfill \break
\hfill \break
\end{center}

% СОДЕРЖАНИЕ
\newpage
    \tableofcontents % Вывод содержания
\newpage
% КОНЕЦ СОДЕРЖАНИЯ 
 
% ЗАДАНИЕ 1
\newpage
\section{Задание №1. Построить конечный автомат, распознающий язык.}
\begin{enumerate}
\item {$L = \{ w \in \{a,b,c\}*$ | $  {|w|_c} = 1 \} $}\\
\ Данный язык включает все слова из букв $\left\{ a, b, c \right\}$, но содержащие только одну букву $c$.  \\
\digraph[scale=1]{task11}{
	rankdir=LR; 
	node [shape=none]; start;
	node [shape=circle]; 1;
	node [shape=doublecircle]; 2;
	start -> 1;
	1 -> 2 [label = "c"];
	1 -> 1 [label = "a,b"];
	2 -> 2 [label = "a,b"];
}

\item {$L = \{ w \in \{a,b\}*$ | $  {|w|_a} \le 2, {|w|_b} \ge 2 \}$}\\ 
\ В данном случае может быть 1 или 2 буквы а и любое количество букв b, начиная с двух\\
\digraph[scale=0.8]{task12}{
	rankdir=LR; 
	node [shape=none]; start;
	node [shape=circle]; 1 2 4 5 7 8;
	node [shape=doublecircle]; 3 6 9;
	start -> 1;
	1 -> 2 [label = "b"];
	1 -> 4 [label = "a"];
	2 -> 3 [label = "b"];
	2 -> 5 [label = "a"];
	3 -> 3 [label = "b"];
	3 -> 6 [label = "a"];

	4 -> 5 [label = "b"];
	4 -> 7 [label = "a"];
	5 -> 6 [label = "b"];
	5 -> 8 [label = "a"];
	6 -> 6 [label = "b"];
	6 -> 9 [label = "a"];

	7 -> 8 [label = "b"];
	8 -> 9 [label = "b"];
	9 -> 9 [label = "b"];
}

\item {$L = \{ w \in \{a,b\}*$ | $  {|w|_a} \ne {|w|_b}  \}$} \\
\ Для данного задания построить автомат нельзя, т.к. для распознавания этого языка требуется запоминать количество символов.\\

\item {$L = \{ w \in \{a,b\}*$ | $  {ww = www} \}$} \\
\ В данном задании язык может состоять только из пустых символов. \\
\digraph[scale=1]{task14}{
	rankdir=LR; 
	node [shape=none]; start;
	node [shape=doublecircle]; 1;
	start -> 1;
}
\end{enumerate}



% ЗАДАНИЕ 2
\section{Задание №2. Построить конечный автомат, используя прямое произведение.}
\begin{enumerate}

\item {$L_1 = \{ w \in \{a,b\}  $ | $  {|w|_a} \ge 2  \wedge   {|w|_b} \ge 2 \} $}\\
\ Построим автомат: 
$L_1_1 = \{ w \in \{a,b\}  $ | $  {|w|_a} \ge 2 \} $ \\
\digraph[scale=1]{task211}{
	rankdir=LR; 
	node [shape=none]; start;
	node [shape=circle]; 1 2;
	node [shape=doublecircle]; 3;
	start -> 1;
	1 -> 2 [label = "a"];
	1 -> 1 [label = "b"];
	2 -> 3 [label = "a"];
	2 -> 2 [label = "b"];
	3 -> 3 [label = "a,b"];
}
\\
\ Построим автомат:
$L_1_2 = \{ w \in \{a,b\}  $ | $  {|w|_b} \ge 2 \} $ \\
\digraph[scale=1]{task212}{
	rankdir=LR; 
	node [shape=none]; start;
	node [shape=circle]; 1 2;
	node [shape=doublecircle]; 3;
	start -> 1;
	1 -> 2 [label = "b"];
	1 -> 1 [label = "a"];
	2 -> 3 [label = "b"];
	2 -> 2 [label = "a"];
	3 -> 3 [label = "a,b"];
} \\
\ Для первого автомата: \\
$ A_{1} = \left\{ w \in \left\{ a, b \right\}^{*} \mid \left| w \right|_{a} \geq 2 \right\} $
$ \Sigma_{A} = \left\{ a, b \right\}$
$ Q_{A} = \left\{ 1, 2, 3 \right\} $
$ s_{A} = \left\{ 1 \right\} $
$ T_{A} = \left\{ 3 \right\}$

\ Для второго автомата: \\
$ B_{1} = \left\{ w \in \left\{ a, b \right\}^{*} \mid \left| w \right|_{b} \geq 2 \right\} $

$ \Sigma_{B} = \left\{ a, b \right\}$
$ Q_{B} = \left\{ 1, 2, 3 \right\} $
$ s_{B} = \left\{ 1 \right\} $
$ T_{B} = \left\{ 3 \right\}$

\ Имеем: \\
$L_{1} = A_{1} \times B_{1}$ \\
$ \Sigma = \Sigma_{A} \cup \Sigma_{B} = \left\{ a, b \right\}$ \\
$ Q = Q_{A} \times Q_{B} = \left\{ 11, 12, 13, 21, 22, 23, 31, 32, 33 \right\}$ \\
$ s = <s_{1}, s_{2}> = \left\{ 11 \right\}$ \\
$ T = T_{A} \times T_{B} = \left\{ 33 \right\}$ \\
\\ \\
\ Построим таблицу состояний: \\
\begin{table}[!htbp]
\rowcolors{2}{white}{blue!10}
\centering
\begin{tabular}{|l|l|l|l|}
\hline
$A$ & $B$ & $a$ & $b$ \\ \hline
1   & 1   & 21  & 12  \\ \hline
1   & 2   & 22  & 13  \\ \hline
1   & 3   & 23  & 13  \\ \hline
2   & 1   & 31  & 22  \\ \hline
2   & 2   & 32  & 23  \\ \hline
2   & 3   & 33  & 23  \\ \hline
3   & 1   & 31  & 32  \\ \hline
3   & 2   & 32  & 33  \\ \hline
3   & 3   & 33  & 33  \\ \hline
\end{tabular}
\end{table}

\ Тогда имеем автомат: \\
\digraph[scale=0.8]{task213}{
	rankdir=LR; 
	node [shape=none]; start;
	node [shape=circle]; 11 12 13 21 22 23 31 32;
	node [shape=doublecircle]; 33;
	start -> 11;
	11 -> 12 [label = "b"];
	11 -> 21 [label = "a"];
	12 -> 13 [label = "b"];
	12 -> 22 [label = "a"];
	13 -> 13 [label = "b"];
	13 -> 23 [label = "a"];

	21 -> 22 [label = "b"];
	21 -> 31 [label = "a"];
	22 -> 23 [label = "b"];
	22 -> 32 [label = "a"];
	23 -> 23 [label = "b"];
	23 -> 33 [label = "a"];

	31 -> 32 [label = "b"];
	31 -> 31 [label = "a"];
	32 -> 33 [label = "b"];
	32 -> 32 [label = "a"];
	33 -> 33 [label = "b"];
}
\\
\item {$L_2 = \{ w \in \{a,b\}*$ | $  {|w|} \ge 3 \wedge {|w|} $ нечётное $ \} $}\\
\ Построим автомат: \\
\digraph[scale=1]{task221}{
	rankdir=LR; 
	node [shape=none]; start;
	node [shape=circle]; 1;
	node [shape=circle]; 2;
	node [shape=circle]; 3;
	node [shape=doublecircle]; 4;
	start -> 1;
	1 -> 2 [label = "a,b"];
	2 -> 3 [label = "a,b"];
	3 -> 4 [label = "a,b"];
	4 -> 4 [label = "a,b"];
}
\\
\ Построим автомат: \\
\digraph[scale=1]{task222}{
	rankdir=LR; 
	node [shape=none]; start;
	node [shape=circle]; 1;
	node [shape=doublecircle]; 2;
	start -> 1;
	1 -> 2 [label = "a,b"];
	2 -> 1 [label = "a,b"];
}
\\
\ Для первого автомата: \\
$ A_{2} = \left\{ w \in \left\{ a, b \right\}^{*} \mid \left| w \right| \geq 3 \right\} $ \\
$ \Sigma_{A} = \left\{ a, b \right\}$ \\
$ Q_{A} = \left\{ 1, 2, 3, 4 \right\} $ \\
$ s_{A} = \left\{ 1 \right\} $ \\
$ T_{A} = \left\{ 4 \right\}$ \\
\\
\ Для второго автомата: \\
$ B_{2} = \left\{ w \in \left\{ a, b \right\}^{*} \mid  \left| w \right| \text{нечетное} \right\} $ \\
$ \Sigma_{B} = \left\{ a, b \right\}$ \\
$ Q_{B} = \left\{ 1, 2 \right\} $ \\
$ s_{B} = \left\{ 1 \right\} $ \\
$ T_{B} = \left\{ 2 \right\}$ \\
\\
\ Тогда имеем: \\
$L_{2} = A_{2} \times B_{2}$ \\
$ \Sigma = \left\{ a, b \right\}$ \\
$ Q = \left\{ 11, 12, 21, 22, 31, 32, 41, 42 \right\}$ \\
$ s = \left\{ 11 \right\}$ \\
$ T = \left\{ 42 \right\}$ \\
\\ \\ \\
\ Построим таблицу состояний: \\
\begin{table}[!htbp]
\rowcolors{2}{white}{blue!10}
\centering
\begin{tabular}{|l|l|l|l|}
\hline
$A$ & $B$ & $a$ & $b$ \\ \hline
1   & 1   & 22  & 22  \\ \hline
1   & 2   & 21  & 21  \\ \hline
2   & 1   & 32  & 32  \\ \hline
2   & 2   & 31  & 31  \\ \hline
3   & 1   & 42  & 42  \\ \hline
3   & 2   & 41  & 41  \\ \hline
4   & 1   & 42  & 42  \\ \hline
4   & 2   & 41  & 41  \\ \hline
\end{tabular}
\end{table}

\ Тогда имеем автомат: \\
\digraph[scale=0.8]{task223}{
	rankdir=LR; 
	node [shape=none]; start;
	node [shape=circle]; 11 12 21 22 31 32 41;
	node [shape=doublecircle]; 42;
	start -> 11;
	11 -> 22 [label = "a,b"];
	12 -> 21 [label = "a,b"];
	21 -> 32 [label = "a,b"];
	22 -> 31 [label = "a,b"];
	31 -> 42 [label = "a,b"];
	32 -> 41 [label = "a,b"];
	41 -> 42 [label = "a,b"];
	42 -> 41 [label = "a,b"];

}

\item {$L_3$ = \{ $w$ \in \{$a,b$\}$*$ $|$  $ {|w|_a}$ чётно  $\wedge$ ${|w|_b}$ кратно трём \} }\\
\ Построим автомат: \\
\digraph[scale=1]{task231}{
	rankdir=LR; 
	node [shape=none]; start;
	node [shape=circle]; 2;
	node [shape=doublecircle]; 1;
	start -> 1;
	1 -> 1 [label = "b"];
	1 -> 2 [label = "a"];
	2 -> 2 [label = "b"];
	2 -> 1 [label = "a"];
}
\ Построим автомат: \\
\digraph[scale=1]{task232}{
	rankdir=LR; 
	node [shape=none]; start;
	node [shape=circle]; 2 3;
	node [shape=doublecircle]; 1;
	start -> 1;
	1 -> 1 [label = "a"];
	1 -> 2 [label = "b"];
	2 -> 2 [label = "a"];
	2 -> 3 [label = "b"];
	3 -> 3 [label = "a"];
	3 -> 1 [label = "b"];
}
\ Для первого автомата: \\
$ A_{3} = \left\{ w \in \left\{ a, b \right\}^{*} \mid \left| w \right|_{a} \text{четное} \right\} $ \\
$ \Sigma_{A} = \left\{ a, b \right\}$ \\
$ Q_{A} = \left\{ 1, 2 \right\} $ \\
$ s_{A} = \left\{ 1 \right\} $ \\
$ T_{A} = \left\{ 1 \right\}$ \\
\\
\ Для второго автомата: \\
$ B_{3} = \left\{ w \in \left\{ a, b \right\}^{*} \mid  \left| w \right|_{b} \text{кратно трем} \right\} $ \\
$ \Sigma_{B} = \left\{ a, b \right\}$ \\
$ Q_{B} = \left\{ 1, 2, 3 \right\} $ \\
$ s_{B} = \left\{ 1 \right\} $ \\
$ T_{B} = \left\{ 1 \right\}$ \\
\ Имеем: \\
$L_{3} = A_{3} \times B_{3}$ \\
$ \Sigma_{3} = \left\{ a, b \right\}$ \\
$ Q_{3} = \left\{ 11, 12, 13, 21, 22, 23 \right\}$ \\
$ s_{3} = \left\{ 11 \right\}$ \\
$ T_{3} = \left\{ 11 \right\}$ \\
\\ \ Построим таблицу состояний: \\
\begin{table}[!htbp]
\rowcolors{2}{white}{blue!10}
\centering
\begin{tabular}{|l|l|l|l|}
\hline
$A$ & $B$ & $a$ & $b$ \\ \hline
1   & 1   & 21  & 12  \\ \hline
1   & 2   & 22  & 13  \\ \hline
1   & 3   & 23  & 11  \\ \hline
2   & 1   & 11  & 22  \\ \hline
2   & 2   & 12  & 23  \\ \hline
2   & 3   & 13  & 21  \\ \hline
\end{tabular}
\end{table}

\ Тогда имеем автомат: \\
\digraph[scale=0.8]{task233}{
	rankdir=LR; 
	node [shape=none]; start;
	node [shape=circle]; 12 13 21 22 23;
	node [shape=doublecircle]; 11;
	start -> 11;
	11 -> 12 [label = "b"];
	11 -> 21 [label = "a"];
	12 -> 13 [label = "b"];
	12 -> 22 [label = "a"];
	13 -> 11 [label = "b"];
	13 -> 23 [label = "a"];

	21 -> 22 [label = "b"];
	21 -> 11 [label = "a"];
	22 -> 23 [label = "b"];
	22 -> 12 [label = "a"];
	23 -> 21 [label = "b"];
	23 -> 13 [label = "a"];
}

\item {$L_4$ = $\overline{L_3}$}\\
\ Данный язык будет распознаяаться автоматом: \\
\digraph[scale=0.8]{task24}{
	rankdir=LR; 
	node [shape=none]; start;
	node [shape=doublecircle]; 12 13 21 22 23;
	node [shape=circle]; 11;
	start -> 11;
	11 -> 12 [label = "b"];
	11 -> 21 [label = "a"];
	12 -> 13 [label = "b"];
	12 -> 22 [label = "a"];
	13 -> 11 [label = "b"];
	13 -> 23 [label = "a"];

	21 -> 22 [label = "b"];
	21 -> 11 [label = "a"];
	22 -> 23 [label = "b"];
	22 -> 12 [label = "a"];
	23 -> 21 [label = "b"];
	23 -> 13 [label = "a"];
}
$ \bar{L_{3}} = \left\{ \Sigma_{3}, Q_{3}, s_{3}, Q_{3}\backslash T_{3}, \delta_{3} \right\} $ \\
$ T_{4} = Q_{3} \backslash T_{3} = \left\{ 12, 13, 21, 22, 23 \right\} $ \\

\item {$L_5 = L_2 \setminus L_3 $}\\
$ L_{5} = L_{2} \backslash L_{3} = L_{2} \cap \bar{L_{3}} = L_{2} \times \bar{L_{3}} $ \\
\ Автомат $ L_{2} $ можно успростить: \\
\digraph[scale=1]{task251}{
	rankdir=LR; 
	node [shape=none]; start;
	node [shape=circle]; 1 2 3;
	node [shape=doublecircle]; 4;
	start -> 1;
	1 -> 2 [label = "a,b"];
	2 -> 3 [label = "a,b"];
	3 -> 4 [label = "a,b"];
	4 -> 3 [label = "a,b"];
}

\ Введем для автомата $ \bar{L_{3}} $ новую нумерацию состояний:\\
\digraph[scale=0.8]{task252}{
	rankdir=LR; 
	node [shape=none]; start;
	node [shape=doublecircle]; 2 3 4 5 6;
	node [shape=circle]; 1;
	start -> 1;
	1 -> 2 [label = "b"];
	1 -> 4 [label = "a"];
	2 -> 3 [label = "b"];
	2 -> 5 [label = "a"];
	3 -> 1 [label = "b"];
	3 -> 6 [label = "a"];

	4 -> 5 [label = "b"];
	4 -> 1 [label = "a"];
	5 -> 6 [label = "b"];
	5 -> 2 [label = "a"];
	6 -> 4 [label = "b"];
	6 -> 3 [label = "a"];
}
\newpage
\ Построим таблицу состояний: \\
\begin{table}[!htbp]
\rowcolors{2}{white}{blue!10}
\centering
\begin{tabular}{|l|l|l|l|}
\hline
$L_{2}$ & $\bar{L_{3}}$ & $a$ & $b$ \\ \hline
1       & 1             & 24  & 22  \\ \hline
1       & 2             & 25  & 23  \\ \hline
1       & 3             & 26  & 21  \\ \hline
1       & 4             & 21  & 25  \\ \hline
1       & 5             & 22  & 26  \\ \hline
1       & 6             & 23  & 24  \\ \hline
2       & 1             & 34  & 32  \\ \hline
2       & 2             & 35  & 33  \\ \hline
2       & 3             & 36  & 31  \\ \hline
2       & 4             & 31  & 35  \\ \hline
2       & 5             & 32  & 36  \\ \hline
2       & 6             & 33  & 34  \\ \hline
3       & 1             & 44  & 42  \\ \hline
3       & 2             & 45  & 43  \\ \hline
3       & 3             & 46  & 41  \\ \hline
3       & 4             & 41  & 45  \\ \hline
3       & 5             & 42  & 46  \\ \hline
3       & 6             & 43  & 44  \\ \hline
4       & 1             & 34  & 32  \\ \hline
4       & 2             & 35  & 33  \\ \hline
4       & 3             & 36  & 31  \\ \hline
4       & 4             & 31  & 35  \\ \hline
4       & 5             & 32  & 36  \\ \hline
4       & 6             & 33  & 34  \\ \hline
\end{tabular}
\end{table} 
\\
\ Получим автомат: \\
\digraph[scale=0.5]{task253}{
	rankdir=LR; 
	node [shape=none]; start;
	node [shape=doublecircle]; 42 43 45 46;
	node [shape=circle]; 11 12 13 14 15 16 21 22 23 24 25 26 31 32 33 34 35 36 41 44;
	start -> 11;
	11 -> 24 [label="a"]
	12 -> 25 [label="a"]
	13 -> 26 [label="a"]
	14 -> 21 [label="a"]
	15 -> 22 [label="a"]
	16 -> 23 [label="a"]
	21 -> 34 [label="a"]
	22 -> 35 [label="a"]
	23 -> 36 [label="a"]
	24 -> 31 [label="a"]
	25 -> 32 [label="a"]
	26 -> 33 [label="a"]
	31 -> 44 [label="a"]
	32 -> 45 [label="a"]
	33 -> 46 [label="a"]
	34 -> 41 [label="a"]
	35 -> 42 [label="a"]
	36 -> 43 [label="a"]
	41 -> 34 [label="a"]
	42 -> 35 [label="a"]
	43 -> 36 [label="a"]
	44 -> 31 [label="a"]
	45 -> 32 [label="a"]
	46 -> 33 [label="a"]
	
	11 -> 22 [label="b"]
	12 -> 23 [label="b"]
	13 -> 21 [label="b"]
	14 -> 25 [label="b"]
	15 -> 26 [label="b"]
	16 -> 24 [label="b"]
	21 -> 32 [label="b"]
	22 -> 33 [label="b"]
	23 -> 31 [label="b"]
	24 -> 35 [label="b"]
	25 -> 36 [label="b"]
	26 -> 34 [label="b"]
	31 -> 42 [label="b"]
	32 -> 43 [label="b"]
	33 -> 41 [label="b"]
	34 -> 45 [label="b"]
	35 -> 46 [label="b"]
	36 -> 44 [label="b"]
	41 -> 32 [label="b"]
	42 -> 33 [label="b"]
	43 -> 31 [label="b"]
	44 -> 35 [label="b"]
	45 -> 36 [label="b"]
	46 -> 34 [label="b"]
}
\ После упрощения получим: \\
\digraph[scale=0.5]{task254}{
	rankdir=LR; 
	node [shape=none]; start;
	node [shape=doublecircle]; 42 43 45 46;
	node [shape=circle]; 11 21 22 24 31 32 33 34 35 36 41 44;
	start -> 11;
	11 -> 24 [label="a"]
	21 -> 34 [label="a"]
	22 -> 35 [label="a"]
	24 -> 31 [label="a"]
	31 -> 44 [label="a"]
	32 -> 45 [label="a"]
	33 -> 46 [label="a"]
	34 -> 41 [label="a"]
	35 -> 42 [label="a"]
	36 -> 43 [label="a"]
	41 -> 34 [label="a"]
	42 -> 35 [label="a"]
	43 -> 36 [label="a"]
	44 -> 31 [label="a"]
	45 -> 32 [label="a"]
	46 -> 33 [label="a"]

	11 -> 22 [label="b"]
	21 -> 32 [label="b"]
	22 -> 33 [label="b"]
	24 -> 35 [label="b"]
	31 -> 42 [label="b"]
	32 -> 43 [label="b"]
	33 -> 41 [label="b"]
	34 -> 45 [label="b"]
	35 -> 46 [label="b"]
	36 -> 44 [label="b"]
	41 -> 32 [label="b"]
	42 -> 33 [label="b"]
	43 -> 31 [label="b"]
	44 -> 35 [label="b"]
	45 -> 36 [label="b"]
	46 -> 34 [label="b"]
}

\end{enumerate}

\end{document}  % КОНЕЦ ДОКУМЕНТА !